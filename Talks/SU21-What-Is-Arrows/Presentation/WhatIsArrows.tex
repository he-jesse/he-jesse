\documentclass{beamer}
\usepackage[utf8]{inputenc}
\usepackage{enumerate, parskip, hyperref}
\usepackage[sorting=none]{biblatex}
\usepackage{tikz-cd}
\urlstyle{same}

\addbibresource{WhatIsArrows.bib}

\usetheme{Antibes}
\usecolortheme{beaver}
\setbeamertemplate{enumerate items}[default]
\setbeamertemplate{bibliography item}[triangle]

\theoremstyle{definition}
\newtheorem*{exercise}{Bonus Exercise}

\title{What is Arrow's Impossibility Theorem?}
\author{Jesse He}
\institute{OSU What Is?}
\date{17 June, 2021}

\begin{document}
    
\frame{\titlepage}

\section{We live in a society}

\begin{frame}
    \frametitle{Definitions}
    
    Let $A$ be a set of $M \geq 3$ alternatives, $A = \{a_1, a_2, \dots, a_M\}$ and let $V = \{1, 2, \dots, N\}$ be a set of $N$ voters.

    \pause
    Denote by $\mathcal{L}(A)$ the set of linear orders (\emph{preference orders}) on $A$.

    \pause
    Each voter $v \in V$ has an associated preference ordering $\leq_v$, and a \emph{preference profile} (or \emph{situation})
    $P = (\leq_1, \dots, \leq_N) \in \mathcal{L}(A)^V$ is an $N$-tuple of preference orders.

    \pause
    A \emph{social welfare function} is a function $F : \mathcal{L}(A)^V \to \mathcal{L}(A)$. We denote the resulting ordering by $\leq_{F(P)}$,
    or just $\leq$.
\end{frame}

\begin{frame}
    \frametitle{More definitions}

    Let $a_i, a_j \in A$.

    \begin{enumerate}
        \item Unanimity (Pareto efficiency)
        \begin{itemize}
            \item<2-> If $a_i <_v a_j$ (that is, $a_i \leq_v a_j$ and $a_j \not\leq_v a_i$) for each $v \in V$, then $a_i \leq_F a_j$. 
        \end{itemize}

        \item Independence of irrelevant alternatives (IIA)
        \begin{itemize}
            \item<3-> If $P = (\leq_v), P' = (\leq'_v) \in \mathcal{L}(A)^V$ such that for all $v \in V$, $a_i \leq_v a_j$
            iff $a_i \leq'_v a_j$, then $a_i \leq_{F(P)} a_j$ iff $a_i \leq_{F(P')} a_j$.
        \end{itemize}
        

        \item Non-dictatorship
        \begin{itemize}
            \item<4-> There is no $v \in V$ such that for all $P \in \mathcal{L}(A)^V$, $a_i <_v a_j$ implies $a_i <_{F(P)} a_j$.
        \end{itemize}
        
    \end{enumerate}

\end{frame}

\begin{frame}
    \frametitle{The statement}

    \begin{theorem}[Arrow's Impossibility Theorem]
        There is no social welfare function that satisfies all three conditions.
    \end{theorem}

\end{frame}

\section{An elementary proof}

\begin{frame}
    \frametitle{Proof}

    Suppose $F$ satisfies unanimity and IIA.
    \pause
    Consider an arbirary $P \in \mathcal{L}(A)^V$ such that $a_i >_v a_j$ for each $v$, and swap the rankings of $a_i$ and $a_j$
    sequentially from $1$ to $N$.

    \pause
    By unanimity, we begin with $a_i > a_j$ and end with $a_i < a_j$.
    \pause
    Denote the first voter whose swap changes the social preference ordering by $v_{ij}$, the $(i,j)$-th \emph{pivotal voter}.
    By IIA, this definition does not depend on $P$.
\end{frame}

\begin{frame}
    \frametitle{Proof}

    Let $P' = (\leq'_v) \in \mathcal{L}(A)^V$ such that $a_i <'_v a_k <'_v a_j$ for $v = 1, 2, \dots, v_{ij} - 1$ and
    $a_k <'_v a_j <'_v a_i$ for $v = v_{ij}, v_{ij} + 1, \dots, N$.
    \pause
    Then $a_k <' a_j <' a_i$ by the definition of $v_{ij}$ and by unanimity.

    \pause
    Let $P'' = (\leq''_v) \in \mathcal{L}(A)^V$ such that $a_i <'' a_j$ and $a_i <'' a_k$ for $v = 1, \dots, v_{ij} - 1$,
    $a_k <_{v_{ij}} a_i <_{v_{ij}} a_j$, and $a_j <'' a_i$ and $a_k <'' a_i$ for $v = v_{ij} + 1, \dots, N$.
    \pause
    Then $a_k  <'' a_i \leq'' a_j$ by the definition of $v_{ij}$ and by IIA.

    \pause
    Then by IIA we see that $a_k <_{v_{ij}} a_j$ implies $a_k < a_j$ for all $i \neq j \neq k$.
\end{frame}

\begin{frame}
    \frametitle{Proof}

    \[
        \begin{array}{c|ccccccc}
            & 1 & \dots & v_{ij}-1 & v_{ij} & v_{ij}+1 & \dots & N \\ \hline
            & a_j & \dots & a_j & a_i & a_i & \dots & a_i \\
           P' & a_k & \dots & a_k & a_j & a_j & \dots & a_j \\
            & a_i & \dots & a_i & a_k & a_k & \dots & a_k \\ \hline
            & \square & \dots & \square & a_j & a_i & \dots & a_i \\
            & a_j\square & \dots & a_j\square & a_i & \square & \dots & \square \\
           P'' & \square & \dots & \square & a_k & a_j\square & \dots & a_j\square \\
            & a_i & \dots & a_i & & \square & \dots & \square \\ \hline
        \end{array}  
    \]

    \pause
    Thus $a_k < a_j$ does not change as long as $a_k <_{v_{ij}} a_j$, so $v_{ij} \leq v_{jk}$. Further, $v_{kj} \leq v_{ij}$.
    \pause
    But $j \neq k$ are arbitrary, so $v_{kj} \leq v_{ij} \leq v_{jk}$ and $v_{jk} \leq v_{kj}$, so $v_{kj} = v_{ij} = v_{jk}$.

    \pause
    Hence $v_{ij}$ is a dictator. \hfill\qedsymbol
\end{frame}

\section{Winning sets}

\begin{frame}
    \frametitle{Winning sets}
    For simplicity, let $A = \{a, b, c\}$, and assume individual preference orders are strict total orders.
    
    \pause
    Consider a situation where $a > b$; by IIA, this depends only on the set $U \subseteq V$ of voters who prefer $a$ to $b$. Call $U$
    a ``winning set'' for $a$ over $b$ if $a > b$ iff $a >_u b$ for each $u \in U$.

    \pause
    We want to be able to find a suitable collection $\mathcal{U} \subseteq \mathcal{P}(V)$ such that if $\{v : a <_v b\} \in \mathcal{U}$,
    then $a < b$. That is, we want $\mathcal{U}$ to be a collection of winning sets.

\end{frame}

\begin{frame}
    \frametitle{Winning sets}

    Suppose we have such a collection $\mathcal{U}$ of winning sets.

    \pause
    Of course, $\varnothing \notin \mathcal{U}$.

    \pause
    Further, voting is decisive: if $U \in \mathcal{U}$, then $U^c \notin \mathcal{U}$, and if $U \notin \mathcal{U}$, then $U^c \in \mathcal{U}$.

    \pause
    Additional votes for an ordering should only help that ordering; if $U \in \mathcal{U}$ and $U \subseteq W \subseteq V$, then $W \in \mathcal{U}$,
    so in particular $V \in \mathcal{U}$.

\end{frame}

\begin{frame}
    \frametitle{Winning sets}
    
    Let $U = \{u : a <_u b\}$ and $W = \{w : b <_w c\}$, and suppose $U, W \in \mathcal{U}$. Then $a < b$ and $b < c$, hence $a < b < c$.
    
    \pause
    Then $U \cap W = \{ v : a <_v b <_v c\} \subseteq \{v : a <_v c\}$, and $\{v : a <_v c\} \in \mathcal{U}$.

    \pause
    It may be that $U \cap W = \{v : a <_v c\}$; as when
    \begin{align*}
        \{v : c <_v a <_v b\} &= U \setminus W, \\
        \{v : b <_v c <_v a\} &= W \setminus U, \\
        \{v : c <_v b <_v a\} &= (U \cup V)^c.
    \end{align*}

    \pause
    Thus it must be in fact that $U \cap W \in \mathcal{U}$.

\end{frame}

\begin{frame}
    \frametitle{Winning sets}

    So we have a collection $\mathcal{U} \subseteq \mathcal{P}(V)$ such that
    \begin{enumerate}
        \item $\varnothing \notin \mathcal{U}$.
        \item If $U \in \mathcal{U}$ and $U \subseteq W \subseteq V$, then $W \in \mathcal{U}$.
        \item If $U, W \in \mathcal{U}$, then $U \cap W \in \mathcal{U}$.
        \item For each $U \subseteq V$, either $U \in \mathcal{U}$ or $U^c \in \mathcal{U}$.
    \end{enumerate}

    \pause
    The collection $\mathcal{U}$ is an \emph{ultrafilter} on $V$!

\end{frame}

\begin{frame}
    \frametitle{Principal and nonprincipal ultrafilters}

    We have already seen an example of an ultrafilter:

    \pause
    Recall our dictator $v_{ij}$, and call them $\hat{v}$.
    \pause
    Any set of voters containing $\hat{v}$ is a winning set, and the winning sets are precisely those containing $\hat{v}$.
    
    \pause
    Our collection $\mathcal{U}$ of winning sets is a \emph{principal ultrafilter}.

    \pause
    What about a \emph{nonprincipal ultrafilter}?

\end{frame}

\begin{frame}
    \frametitle{An alternate perspective}

    Armed with this revelation, Arrow's theorem follows from the statement:

    \pause
    \begin{theorem}
        There are no nonprincipal ultrafilters on finite sets.
    \end{theorem}

    \pause
    We could in fact create a nondictatorial Pareto IIA social welfare function by constructing a nonprincipal ultrafilter,
    provided we have infinite voters and the Axiom of Choice.

\end{frame}

\section{Is this Łoś's?}

\begin{frame}
    \frametitle{Ultraproducts}

    It turns out that the social choice function $F$ is an \emph{ultraproduct}:

    \pause
    Given an index set $I$, a family of ``structures'' (in our case, ordered sets) $(M_i)_{i \in I}$, and an ultrafilter $\mathcal{U}$ on $I$,
    we define on $\prod_{i \in I} M_i$ the equivalence relation $a \sim b$ if $\{i \in I : a_i = b_i\} \in \mathcal{U}$. Then the \emph{ultraproduct} is
    the quotient structure $\prod_{i \in I} M_i / \sim$, sometimes written $\prod_{i \in I} M_i / \mathcal{U}$.

\end{frame}

\begin{frame}
    \frametitle{Łoś's theorem}

    \pause
    \begin{theorem}[Łoś's theorem]
        Let $\sigma$ be a signature, $\mathcal{U}$ an ultrafilter on a set $I$, and let $(M_i)_{i \in I}$ be a family of $\sigma$-structures.
        Let $M = \prod_{i \in I} M_i / \mathcal{U}$. Then for each $a^1, \dots, a^n \in \prod_{i \in I} M_i$, where $a^k = (a^k_i)_{i \in I}$,
        and every $\sigma$-formula $\phi$,
        \pause
        \[
            M \models \phi[[a^1], \dots, [a^n]] \iff \{i \in I : M_i \models \phi[a_i^1, \dots, a_i^n]\} \in \mathcal{U}.
        \]
    \end{theorem}
    \pause
    \begin{proof}
        \pause
        Bonus Exercise.
    \end{proof}

\end{frame}

\section{Secret Hitler}

\begin{frame}
    \frametitle{A Big Black Box}

    \begin{theorem}[Uniqueness of winning ultrafilters]
        \begin{enumerate}
            \pause
            \item For each Pareto IIA social welfare function $F$ there is a unique ultrafilter $\mathcal{U}_F$ such that for each $U \in \mathcal{U}_F$,
            each $a, b \in A$, and any preference profile $P$, if $\{u : a \leq_{u} b\} \in \mathcal{U}_F$ then $a \leq_{F(P)} b$.
            \pause
            \item For each ultrafilter $\mathcal{U}_F$ on $V$ there exists such an $F$.
        \end{enumerate}
    \end{theorem}

\end{frame}

\begin{frame}
    \frametitle{Some more black boxes}
    
    \pause
    \begin{fact}[Fact 1]
        For any atomless measure space $(X, \mathcal{M}, \mu)$, for all $\varepsilon > 0$, there is a finite partition $X = \bigsqcup_{i=1}^r C_i$,
        such that each $C_i \in \mathcal{M}$  and $\mu(C_i) < \varepsilon$.
    \end{fact}

    \pause
    \begin{fact}[Fact 2]
        Let $\mathcal{U}$ be an ultrafilter on a set $X$. For any finite partition $X = \bigsqcup_{i=1}^r C_i$, exactly one $C_i \in \mathcal{U}$.
    \end{fact}

\end{frame}

\begin{frame}
    \frametitle{Arbitrarily small coalitions}

    In the case where $V$ is infinite, consider an atomless measure space $(V, \mathcal{V}, \lambda)$.

    \pause
    \begin{theorem}
        For any $\varepsilon > 0$ there exists $C \in \mathcal{V}$, $C \neq \varnothing$, $\lambda(C) < \varepsilon$ such that for any $a, b \in A$
        and any situation $P$, if $\{v : a \leq_c b\} \subseteq C$ then $a \leq_{F(P)} b$.
    \end{theorem}

    \pause
    \begin{proof}
        Apply the first fact to $(V, \mathcal{V}, \lambda)$. Then we have a partition $X = \bigsqcup_{i=1}^r C_i$ such that each $C_i \in \mathcal{M}$
        and $\mu(C_i) < \varepsilon$. Take the ultrafilter $\mathcal{U}_F$ guaranteed by the uniqueness of winning ultrafilters and apply the second fact.
    \end{proof}

\end{frame}

\begin{frame}
    \frametitle{Stone-\v{C}ech electorate}

    Consider $V$ as a discrete topological space and suppose the set $A$ of alternatives is finite. Then $\mathcal{L}(A)$ is finite, so it is also
    a discrete topological space.

    \pause
    We have the following commutative diagram:
    \begin{center}
        \begin{tikzcd}[ampersand replacement=\&]
            V \arrow[r, "i_V", hook] \arrow[rd, "P"'] \& \mathit{\beta V} \arrow[d, "\exists ! \mathit{\beta P}", dashed] \\
            \& \mathcal{L}(A)                                        
        \end{tikzcd}
    \end{center}
    \pause
    This is the universal property for $\mathit{\beta V}$, the Stone-\v{C}ech compactification of $V$.

    \pause
    Note that $V$ is dense in $\mathit{\beta V}$.

\end{frame}

\begin{frame}
    \frametitle{One more black box}

    \begin{fact}
        Let $\mathcal{U}$ be an ultrafilter on a topological space $X$. Then $\mathcal{U}$ has a (unique) limit in $\mathit{\beta X}$.
    \end{fact}

\end{frame}

\begin{frame}
    \frametitle{A hidden dictator}

    \begin{theorem}
        There exists a point $\hat{v} \in \mathit{\beta V}$ such that for all $a, b \in A$ and $P \in \mathcal{L}(A)^V$, if $a \leq_{\hat{v}} b$
        then $a \leq_{F(P)} b$.
    \end{theorem}

    \begin{proof}
        \pause
        We know that there is a unique winning ultrafilter $\mathcal{U}_F$.

        \pause
        By the previous fact, $\mathcal{U}_F$ has a unique limit $\hat{v} \in \mathit{\beta V}$.
        
        \pause
        Then $\mathit{\beta P}(\hat{v}) = \mathcal{U}_F\lim_v P(v)$. Since $\mathcal{L}(A)$ is discrete, there is some
        $U \in \mathcal{U}_F$ with $P(U) = \mathit{\beta P}(\hat{v})$.
        
        \pause
        By the properties of $\mathcal{U}_F$, $U$ is a winning set, and hence $\hat{v}$ dictates $F$.
    \end{proof}

\end{frame}

\section{References}

\begin{frame}
    \frametitle{References}

    \nocite{*}
    \printbibliography

\end{frame}

\end{document}