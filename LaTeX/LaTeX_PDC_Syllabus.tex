\documentclass{article}
\usepackage{sectsty}
\usepackage[utf8]{inputenc}
\sectionfont{\fontsize{12}{10}\selectfont}

\newcounter{week}
\newenvironment{week}[1][]{\refstepcounter{week}\par\medskip\noindent%
    \textbf{Week~\theweek. #1} \rmfamily}{\medskip}

\title{Basic \LaTeX: An Introduction to Typesetting Using \LaTeX}
\author{Jesse He}
\date{Spring 2020}

\begin{document}

\maketitle

\section{Instructor Information}
Name: Jesse He \\
Email: he.1528@osu.edu \\
Phone: 419-378-5584

\section{Course Description}
This course will introduce you to the typesetting and document preparation tool \LaTeX.
We will go over some basic typesetting tasks, some important concepts, and a few use cases, as well as some more advanced topics.
You will also learn how to search for and read \LaTeX documentation for packages you may find useful.

\section{Materials}
You will need a tablet or laptop you are comfortable typing on.

\section{Attendance and Participation}
Be prepared and attend every class.
If you cannot make a meeting please contact me ahead of time and we will work something out.
Otherwise, absences will result in a point deduction.
STEP credit is based on attendance, but I encourage you to participate and do assignments so that you have practice using \LaTeX.

\section{Assignments}
My goal for you is for you to prepare a well-formatted document on a topic of your choice by the end of this program.
Each week will have a small assignment for you to do either during the meeting time or outside of class.
The purpose of these assignments is to keep you on track for your final document.

\section{Grading}
\begin{tabular}{ll}
    Attendance/Participation: & 35 (5 per class)\\
    Homework: & 30 (5 per assignment)\\
    Final Document: & 35
\end{tabular}

\section{Topics}
\begin{week}
    What is \LaTeX?
    \begin{enumerate}
        \item Why use \LaTeX?
        \item Creating an Overleaf account or installing a local distribution
    \end{enumerate}
\end{week}
\begin{week}
    Basic typesetting in \LaTeX
    \begin{enumerate}
        \item Paragraphs and newlines
        \item Bold, italics, and underlined text
        \item Pictures, figures, and captions
        \item Lists and tables
        \item Document layout and sections
    \end{enumerate}
\end{week}
\begin{week}
    Packages, commands, and environments
    \begin{enumerate}
        \item The document preamble
        \item Common packages
        \item Environments and commands
        \item Creating your own environments and commands
    \end{enumerate}
\end{week}
\begin{week}
    Use case: mathematical, scientific, and technical writing
    \begin{enumerate}
        \item Math: inline and display math, \texttt{equation} and \texttt{align} environments
        \item Math: packages \texttt{amsthm}, \texttt{amsmath}, and \texttt{amssymb}
        \item Computer Science: packages \texttt{listings} and \texttt{minted}
        \item Statistics and Data Analytics: package \texttt{knitr}
        \item Chemistry: package \texttt{chemfig}
    \end{enumerate}
\end{week}
\begin{week}
    Use case: bibliographies and references
    \begin{enumerate}
        \item Package \texttt{biblatex}
        \item Packages \texttt{bibtex} and \texttt{natlib}
    \end{enumerate}
\end{week}
\begin{week}
    Use case: making a presentation
    \begin{enumerate}
        \item Package \texttt{beamer}
        \item Package \texttt{powerdot}
        \item Package \texttt{poster}
    \end{enumerate}
\end{week}
\begin{week}
    Using TikZ
    \begin{enumerate}
        \item Points, lines, paths
        \item Geometric shapes
        \item Diagrams
        \item Subpackages
    \end{enumerate}
\end{week}

\end{document}