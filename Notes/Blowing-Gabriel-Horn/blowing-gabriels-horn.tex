\documentclass{article}
\usepackage[english]{babel}
\usepackage[utf8]{inputenc}
\usepackage[margin=1.0in]{geometry}
\usepackage{enumerate, float, caption, subcaption, nicefrac, sectsty, mathtools, parskip, fancyhdr, amsmath, amsthm, amssymb, tikz, tikz-cd}
\usetikzlibrary{positioning}
\usepackage[en-GB]{datetime2}
\DTMlangsetup[en-GB]{ord=omit}
\usepackage[shortlabels]{enumitem}
\setlist[enumerate,1]{label={(\alph*)}}
\usepackage[unicode]{hyperref}
\hypersetup{
    colorlinks=true,
    linkcolor=blue,
    filecolor=red,
    urlcolor=blue,
}

% \newcommand{\C}{\mathbb{C}}
\newcommand{\R}{\mathbf{R}}
\newcommand{\Q}{\mathbf{Q}}
\newcommand{\Z}{\mathbf{Z}}
\newcommand{\N}{\mathbf{N}}
\newcommand{\F}{\mathbb{F}}
\DeclareMathOperator{\Mat}{Mat}
\DeclareMathOperator{\End}{End}
\DeclareMathOperator{\Hom}{Hom}
\DeclareMathOperator{\Id}{Id}
\DeclareMathOperator{\image}{image}
\DeclareMathOperator{\Imag}{Imag}
\DeclareMathOperator{\rank}{rank}
\DeclareMathOperator{\nullity}{nullity}
\DeclareMathOperator{\trace}{tr}
\DeclareMathOperator{\adj}{adj}
\DeclareMathOperator{\Spec}{Spec}
\DeclareMathOperator{\sla}{\mathfrak{sl}}
\DeclareMathOperator{\Ai}{Ai}
\DeclareMathOperator{\Bi}{Bi}
\DeclareMathOperator{\pf}{pf}
\DeclareMathOperator{\lcm}{lcm}
\DeclareMathOperator{\Ker}{Ker}
\DeclareMathOperator{\coker}{coker}
\DeclareMathOperator{\Aut}{Aut}
\DeclareMathOperator{\Inn}{Inn}
\DeclareMathOperator{\Out}{Out}
\DeclareMathOperator{\rad}{rad}
\DeclareMathOperator{\Nil}{Nil}
\DeclareMathOperator{\Ann}{Ann}
\DeclareMathOperator{\Sym}{Sym}
\DeclareMathOperator{\Tor}{Tor}
\DeclareMathOperator{\sign}{sign}
\DeclareMathOperator{\Gal}{Gal}
\DeclareMathOperator{\Fix}{Fix}
\DeclareMathOperator{\Hol}{Hol}
\newcommand{\Syl}[2]{\operatorname{Syl}_{#1}(#2)}
\newcommand{\norm}[1]{\left\lVert#1\right\rVert}
\newcommand{\transpose}{\intercal}
\newcommand{\pres}[2]{\langle #1 \mid #2 \rangle}

\pagestyle{fancy}

\sectionfont{\fontsize{12}{15}\selectfont}
\subsectionfont{\fontsize{10}{12}\selectfont}

\theoremstyle{definition}
\newtheorem{theorem}{Theorem}[subsection]
\newtheorem{definition}[theorem]{Definition}
\newtheorem{proposition}[theorem]{Proposition}
\newtheorem{lemma}[theorem]{Lemma}
\newtheorem{corollary}[theorem]{Corollary}
\newtheorem{example}[theorem]{Example}
\newtheorem*{remark}{Remark}

\lhead{Jesse He}
\chead{\textbf{On Blowing Gabriel's Horn}}
\rhead{\today}

\begin{document}
    
\section{Introduction.}

We are interested in the question of blowing Gabriel's horn,
ignoring the practical difficulties of procuring an infinite amount
of brass from which to fashion an infinitely large horn; in finding an
infinitesimally small horn player to blow into the horn; in placing
said infinitesimally small horn player infinitely far away so that they may
reach the mouthpiece of the horn; and in sound travelling an infinite
distance at finite speed.

\subsection{Preliminaries.}

We will begin with the following background from mathematics and physics.

\subsubsection{Gabriel's Horn.}

\emph{Gabriel's horn} (also \emph{Torricelli's trumpet}) is the solid
of rotation given by rotating the graph of $y = 1/x$ about the
$x$ axis for $x \geq 1$, first described by the Italian mathematician
Evangelista Torricelli's 1643 paper \emph{De solido hyperbolico acuto}\footnote{``Gabriel's Horn,'' \emph{Wikipedia, the Free Encyclopedia}.
At \url{https://en.wikipedia.org/wiki/Gabriel's_Horn}}.

It is well-known to students of integral calculus that Gabriel's horn
has finite volume but infinite surface area; indeed, the volume of
Gabriel's horn is given by
\[
    V = \pi \int_1^\infty \left(\frac{1}{x}\right)^2 \,dx
    = \lim_{a \to \infty} \pi\left(1 - \frac{1}{a}\right)
    = \pi
\]
while the surface area is given by
\[
    A = 2\pi \int_1^\infty \frac{1}{x}\sqrt{1 + \left(-\frac{1}{x^2}\right)^2} \,dx
    = \infty.
\]

\subsubsection{Webster's Horn Equation.}

The behavior of an acoustic wave in a horn with varying cross-sectional area $S(x)$
can be modeled by \emph{Webster's horn equation}\footnote{For a full derivation, see Section 4.2.3 of \emph{Euphonics: The Science of Musical Instruments}
by Jim Woodhouse at \url{https://euphonics.org/}}
\begin{equation}
    \frac{\partial}{\partial x} \left( S(x)\frac{\partial p}{\partial x} \right)
    = \frac{S}{c^2} \frac{\partial^2 p}{\partial t^2}
\end{equation}
where $c$ is the speed of sound and $p = p(x,t)$ is the \emph{sound pressure}, the
variation in pressure from the standard pressure $p_0 = 14$psi at length $x$ and time $t$.

\section{Blowing the Horn.}

\subsection{Can You Blow the Horn?}

Since the cross-section of the horn at $x$ is a circle of radius $r = 1/x$,
we have
\[
    S(x) = \pi r^2 = \pi \frac{1}{x^2}
\]
and since we are interested in a harmonic sound wave, we let $\omega$ be the
fundamental frequency of the the sound wave and so we may write Equation (1) as
\[
    \frac{d^2 \psi}{dx^2}
    + \left(\frac{\omega^2}{c^2} - \frac{1}{r}\frac{d^2 r}{dx^2}\right)\psi = 0.
\]

Since Webster's horn equation describes the pressure from the mouthpiece of the horn to
to the bell, it will be necessary to make the substitution $u = a - x$ and consider
finite sections of the horn on $[1,a]$ before taking the limit as $a \to \infty$.
Performing this change of variables yields
\begin{equation}
    \frac{d^2 \psi}{du^2} + \left(\frac{\omega^2}{c^2} - \frac{2}{u^2}\right)\psi = 0.
\end{equation}
Now, since the term in parentheses in Equation 2 is negative for $u$ small enough,
it must be that any modes in the horn will fade before reaching the mouth of the horn
and we can conclude, perhaps unsurprisingly, that were one to blow the horn, there would be
no sound at the bell.

\subsection{But What If You Could?}

We have been thus far undeterred by the physical realities of horn playing,
so there is no reason to be so deterred now.

For simplicity, suppose $\omega^2/c^2 = 1$. Then the solution to this linear
second-order ordinary differential equation is
\[
    \psi(u) = \sqrt{\frac{2}{\pi}}
    \left(k_1\left(\frac{\sin u}{u} - \cos u\right)
    - k_2\left(\sin u + \frac{\cos u}{u}\right)\right)
\]
which yields the pressure function
\[
    p(u) = \psi(u)S(u)^{-1/2} = \frac{\psi(u)}{\sqrt{\pi}(a-u)}.
\]
Now, $u$ is only defined for $u < a-1$, so the singularity at $u = a$ is of no consequence.

We can elimiate the singularity at $u = 0$ if one of $k_1$ or $k_2$ is 0;
for simplicity, suppose $k_1 = 1$ and suppose $k_2 = 0$ so that $p(0)$ is finite,
since our infinitely small horn player may have difficulty producing an infinite
pressure wave at the mouthpiece, and even if they were able to such an embochure
would certainly be exhausting to maintain. This assumption will turn out to
have little effect on the final result of the calculation far from the mouthpiece.

We finally consider the simplified pressure function
\[
    p(u) = \frac{\sqrt{\frac{2}{\pi}}\left(\frac{\sin u}{u} - \cos u\right)}
    {\sqrt{\pi}(a-u)}.
\]
At $u = a-1$, the very end of the horn, we have
\[
    p(a-1) = \frac{\sqrt{2}}{\pi}\left(\frac{\sin(1-a)}{1-a} - \cos(1-a)\right),
\]
which is bounded. Thus even taking $a \to \infty$, the sound would not be infinite.

\end{document}